\nonstopmode{}
\documentclass[a4paper]{book}
\usepackage[times,inconsolata,hyper]{Rd}
\usepackage{makeidx}
\usepackage[utf8]{inputenc} % @SET ENCODING@
% \usepackage{graphicx} % @USE GRAPHICX@
\makeindex{}
\begin{document}
\chapter*{}
\begin{center}
{\textbf{\huge Package `drughelper'}}
\par\bigskip{\large \today}
\end{center}
\inputencoding{utf8}
\ifthenelse{\boolean{Rd@use@hyper}}{\hypersetup{pdftitle = {drughelper: Drug Identification and Drug Name Correction with Automatic Actualisations}}}{}\begin{description}
\raggedright{}
\item[Type]\AsIs{Package}
\item[Title]\AsIs{Drug Identification and Drug Name Correction with Automatic
Actualisations}
\item[Version]\AsIs{0.1.0}
\item[Author]\AsIs{Javier Garcia}
\item[Maintainer]\AsIs{Fernando Carazo }\email{fcarazo@tecnun.es}\AsIs{}
\item[Description]\AsIs{Package DrugHelper is a function which identifies and corrects the names a set of drugs between clinical phases 1-4 and that periodically actualize the available drugs and its synonyms.}
\item[License]\AsIs{MIT + file LICENSE}
\item[Imports]\AsIs{progress}
\item[Depends]\AsIs{progress}
\item[Suggests]\AsIs{knitr, rmarkdown,}
\item[Encoding]\AsIs{UTF-8}
\item[RoxygenNote]\AsIs{7.1.1}
\item[VignetteBuilder]\AsIs{knitr, rmarkdown}
\item[NeedsCompilation]\AsIs{no}
\end{description}
\Rdcontents{\R{} topics documented:}
\inputencoding{utf8}
\HeaderA{checkDrugSynonym}{Check for drug synonyms}{checkDrugSynonym}
%
\begin{Description}\relax
Given an input of drug synonyms, check if those drugs are approved and find a proper more used synonym
\end{Description}
%
\begin{Usage}
\begin{verbatim}
checkDrugSynonym(drugVector)
\end{verbatim}
\end{Usage}
%
\begin{Arguments}
\begin{ldescription}
\item[\code{drugVector}] A string vector of undefined length, with Drug names
\end{ldescription}
\end{Arguments}
%
\begin{Value}
A dataframe containing: the input drug name, if it is approved or not, drughelper ID and a proper-more used synonym.
\end{Value}
%
\begin{Examples}
\begin{ExampleCode}
checkDrugSynonym(c("Procaine", "Furazosin", "Embelin", "NotADrug"))
\end{ExampleCode}
\end{Examples}
\inputencoding{utf8}
\HeaderA{downloadAbsentFile}{Download data from Chembl}{downloadAbsentFile}
%
\begin{Description}\relax
If it has not been downloaded yet, downloads data of drugs
\end{Description}
%
\begin{Usage}
\begin{verbatim}
downloadAbsentFile(dir = tempdir())
\end{verbatim}
\end{Usage}
%
\begin{Arguments}
\begin{ldescription}
\item[\code{dir}] Name of the directory where data is downloaded
\end{ldescription}
\end{Arguments}
\inputencoding{utf8}
\HeaderA{formattingDrugName}{Normalization of drug names}{formattingDrugName}
%
\begin{Description}\relax
This function corrects and capitalizes the names of a given vector of drugs.
\end{Description}
%
\begin{Usage}
\begin{verbatim}
formattingDrugName(DrugName)
\end{verbatim}
\end{Usage}
%
\begin{Arguments}
\begin{ldescription}
\item[\code{DrugName}] A string with the name of the drug to be normalized
\end{ldescription}
\end{Arguments}
%
\begin{Examples}
\begin{ExampleCode}
formattingDrugName("morphine")
\end{ExampleCode}
\end{Examples}
\printindex{}
\end{document}
